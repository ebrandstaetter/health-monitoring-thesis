\section{Overview}
\setauthor{Elias Brandstätter}

Allgemeines und Bild über den Tech-Stack...

% Zoe Frontend Part
\section{Frontend}
\setauthor{Zoe Emily Öllinger}

\subsection{Vue.js}
\setauthor{Zoe Emily Öllinger}

\subsection{...}
\setauthor{Zoe Emily Öllinger}
% Zoe Frontend Part


\section{Backend}
\setauthor{Elias Brandstätter}

\subsection{Python}
\setauthor{Elias Brandstätter}
Das Backend wurde mit Python 3 implementiert. Diese Entscheidung für Python basiert einerseits auf bestehende Coding-Guidelines und andererseits auf der hohen Lesbarkeit und Wartbarkeit der Sprache. Außerdem ermöglicht Python eine klare, strukturierte Entwicklung von REST-Schnittstellen und bietet durch sein umfangreiches Ökosystem eine große Auswahl an Bibliotheken für Webentwicklung, Datenverarbeitung und Cloud-Integration. Besonders im Kontext von Datenabfragen und API-Integration bietet Phyton eine effiziente und gut dokumentierte Grundlage.

\subsection{FastAPI}
\setauthor{Elias Brandstätter}


\subsection{Uvicorn}
\setauthor{Elias Brandstätter}

\subsection{Pydantic}
\setauthor{Elias Brandstätter}


\section{Cloud \& Datenerhaltung}
\setauthor{Elias Brandstätter}

\subsection{Google BigQuery}
\setauthor{Elias Brandstätter}

\subsection{Google Cloud Platform}
\setauthor{Elias Brandstätter}

\subsection{SQL}
\setauthor{Elias Brandstätter}


\section{Externe Schnittstellen}
\setauthor{Elias Brandstätter}

\subsection{Asana API}
\setauthor{Elias Brandstätter}

\subsection{HubSpot API}
\setauthor{Elias Brandstätter}

\subsection{Google Play Developer API}
\setauthor{Elias Brandstätter}

\subsection{App Store Connect API}
\setauthor{Elias Brandstätter}